% !TEX TS-program = lualatex
% !TEX encoding = UTF-8 Unicode

\documentclass[
  fontset = none,
  punct   = kaiming,
]{ctexart}
\usepackage{fontspec}
\usepackage{fancyhdr}
\usepackage{luatexja-ruby}
\usepackage{minted}
\usepackage[
  margin=1in,
  headheight=14pt,
]{geometry}
\usepackage[dvipsnames]{xcolor}
\usepackage[
  colorlinks,
  linkcolor = BrickRed,
  citecolor = Green,
  filecolor = Mulberry,
  urlcolor  = NavyBlue,
  menucolor = BrickRed,
  runcolor  = Mulberry,
]{hyperref}
\usepackage[
  backend  = biber,
  style    = caspervector,
  seconds,
  utf8,
  backref,
]{biblatex}
\usepackage[
  perpage,
  hang,
  flushmargin,
]{footmisc}
\usepackage{footnotebackref}
\usepackage{pifont}
\usepackage{graphicx}

\graphicspath{{Images/}}

\renewcommand{\emph}[1]{\textbf{#1}}

% 参考文献
\addbibresource{ref.bib}

% 代码块
\renewcommand{\theFancyVerbLine}{%
  \ttfamily {%
    \oldstylenums{\arabic{FancyVerbLine}}
  }
}
\setminted{
  mathescape,
  linenos,
  frame    = lines,
  framesep = 2mm,
}

% 字体
\setmainfont{SF Pro Text}[
  ItalicFont = New York Small Regular Italic
]
\setsansfont{SF Pro Text}[
  ItalicFont = New York Small Regular Italic
]
\setCJKmainfont{Hiragino Sans GB}[
  ItalicFont = STKaiti,
]
\setCJKsansfont{Hiragino Sans GB}[
  ItalicFont = STKaiti,
]
\setCJKmonofont{Hiragino Sans GB}
\newcommand{\emoji}[1]{
  {\setmainfont{Apple Color Emoji}[Renderer=Harfbuzz]{#1}}
}

\newCJKfontfamily\song{Songti SC Light}[
  BoldFont   = Songti SC Bold,
  ItalicFont = STKaiti,
]

\setmonofont[
  ItalicFont     = JetBrains Mono Italic,
  BoldFont       = JetBrains Mono Bold,
  BoldItalicFont = JetBrains Mono Bold Italic,
  Contextuals    = Alternate,
]{JetBrains Mono}

\newfontfamily\lmmono{Latin Modern Mono}
\newCJKfontfamily\sans{Hiragino Sans GB}
\newCJKfontfamily\gothic{Hiragino Sans}
\newCJKfontfamily\mincho{Hiragino Mincho ProN}

% 链接样式
\urlstyle{same}
\let\oldurl\url
\renewcommand{\url}[1]{%
{\lmmono\oldurl{#1}}
}

% Footnote
%% Circled num
\renewcommand\thefootnote{
  \textcolor{black}\ding{\numexpr171+\value{footnote}}
}

%% Backref
\makeatletter
\LetLtxMacro{\BHFN@Old@footnotemark}{\@footnotemark}

\renewcommand*{\@footnotemark}{%
    \refstepcounter{BackrefHyperFootnoteCounter}%
    \xdef\BackrefFootnoteTag{bhfn:\theBackrefHyperFootnoteCounter}%
    \label{\BackrefFootnoteTag}%
    \BHFN@Old@footnotemark
}
\makeatother

% CTex 设置
\ctexset{
  section={
    name={第,节},
    number=\arabic{section},
  }
}

% 页眉
\fancyhf{}
\lhead{\textnormal{\rightmark}}
\rhead{--\ \thepage\ --}
\pagestyle{fancy}
\renewcommand\sectionmark[1]{%
  \markright{\CTEXifname{\CTEXthesection}{}\quad{}#1}}


% 页脚
\fancyfoot[C]{\thepage}

\author{Colerar}
\date{\today}
\title{Lua\LaTeX{}通用模板}

\begin{document}

% 标题
\maketitle

% 目录
{
  \hypersetup{linkcolor=black}
  \tableofcontents
}

\newpage
\sans
\section{汉语}

汉语,日韩称中国语,书写又可称汉文、中文、华文、唐文,语言又称“华语”、唐话、中国话等,
汉语是联合国官方语言之一,属汉藏语系的分析语,具有声调。汉语的文字系统——
汉字是一种意音文字,表意的同时也具一定的表音功能。汉语包含书面语以及口语两部分,
古代书面汉语称为文言文,现代书面汉语一般指使用现代标准汉语语法、词汇的中文通行文体
(又称白话文)。目前全球有六分之一人口使用汉语作为母语。汉语口语主要分为官话、吴语、
闽语、粤语、湘语、客家语和赣语等七种;它们的语言学归属在西方语言学界存在争议,
或被认为是独立的语言,或被认为是汉语方言。


\newpage
\section[日本語]{\gothic 日本語 }

\mincho

\emph{\ltjruby{日本|語}{にほん|ご}}は、主に日本国内や日本人同士の間で使われている
言語である。日本は法令によって公用語を規定していないが、法令その他の公用文は全て
日本語で記述され、各種法令において日本語を用いることが定められ、学校教育においては
「国語」として教えられるなど、事実上、唯一の公用語となっている。


\newpage
\sans
\section{汉字}

\emph{汉字},在中国亦称中文字、国字、唐字、方块字,是汉字文化圈广泛使用的一种文字,
是世界上独有的一种指示会意文字体系,也是世界上唯一仍被广泛使用并高度发展的语素文字,
为中国上古时代的华夏族人所发明创制,其字体也历经过长久改进及演变。目前确切历史,
可追溯至约公元前1300年商朝的甲骨文、籀文、金文,再到春秋战国与秦朝的籀文、小篆,
发展至汉朝隶变,产生隶书、草书以及楷书\footnote{以及派生的行书},至唐代楷化为今日所用的
手写字体标准——正楷,也是今日普遍使用的现代汉字。汉字在古文中只称“字”,少数民族为区别
而称“汉字”,指汉人使用的文字,后者称法在近代才开始通用。

\newpage
\gothic
\section{仮名} \label{sec:kana}

\emph{\ltjruby{仮名}{かな}}とは、漢字をもとにして日本で作られた文字のこと。
現在一般には平仮名と片仮名のことを指す。表音文字の一種であり、基本的に1字が1音節を
あらわす音節文字に分類される。漢字に対して\ltjruby{和字}{わじ}ともいう。ただし和字は和製漢字を
意味することもある。

\newpage
\sans
\section{章节}

\subsection{子节}
\subsection{子节}
\subsubsection{子子节}

\paragraph{段落} 这是一个段落。
\subparagraph{子段落} 这是一个子段落。

\newpage
\section{代码块}

\begin{minted}{haskell}
import Control.Monad.State
fib n = flip evalState (0, 1) $ do
  forM [0..(n - 1)] $ \_ -> do
    (a, b) <- get
    put (b, a + b)
  (a, b) <- get
  return a
\end{minted}

下方代码块被高亮了:

\begin{minted}[
  highlightlines={6-9}
]{kotlin}
suspend fun main(args: Array<String>) {
  when (val result = SorapointaMain().main(args)) {
      is CommandResult.Success -> {
          exitProcess(0)
      }
      is CommandResult.Error -> {
          println(result.userMessage)
          exitProcess(1)
      }
  }
}
\end{minted}


\newpage
\section{图片}

图 \ref{fig:example} 来自 Pexels,为版权\copyright{}作品,不按照本项目授权分发。

\begin{figure}[htbp]
  \centering
  \includegraphics[scale=0.2]{pexels-1509534.jpg}
  \caption{取自 Pexels \url{https://www.pexels.com/photo/1509534/}}
  \label{fig:example}
\end{figure}

\begin{figure}[htbp]
\centering
\includegraphics[scale=0.2]{pexels-1509534.jpg}
\includegraphics[scale=0.2]{pexels-1509534.jpg}
\\[\smallskipamount]
\includegraphics[scale=0.2]{pexels-1509534.jpg}
\includegraphics[scale=0.2]{pexels-1509534.jpg}
\caption{多个图像并排}
\end{figure}


\newpage
\section{其他}

\sans

\begin{description}
  \item[链接] \url{https://www.example.com}
  \item[带描述的链接] \href{https://www.ctan.org}{CTAN}
  \item[跳转链接]
    \hyperref[sec:kana]{点我跳转至{\gothic{}仮名}}
  \item[表情符号] \emoji{😶‍🌫️😂👍👪}
  \item[参考文献] Lorem ipsum \ldots \autocite{greenwade93}
  \item[中文参考文献] 复活币的锻造方法。 \autocite{bili_lex}
  \item[脚注中的链接\footnotemark] 参见下方
  \footnotetext{\href{https://example.org}{示例}}
  \item[脚注中的多行文本\footnotemark] 参见下方
  \footnotetext{%
    多行文本也能够轻易对齐。\\
    这个功能使用 \texttt{footmisc} 宏包。
  }
\end{description}

% 参考文献
\printbibliography[heading=bibintoc]

\end{document}
